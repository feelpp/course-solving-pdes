\begin{frame}[standout]{Exercising with the Toolboxes}
  \url{https://github.com/feelpp/course-solving-pdes/examples}
\end{frame}

\section{Toolbox  Framework}

\begin{frame}{Model Properties: Why?}
  We have a nice DSEL that allows to write variational formulations,
  we have the cfg files as well as Ginac. However everyone comes up
  with a different solution, we need to have a more unified way to
  describe our models.

  \centering
  \vfill\vspace{1em}\usebeamerfont{section title}{\scshape Introduce
    Model Properties Framework}\vfill
\end{frame}

\begin{frame}{Model Properties: What?}

What are model properties?
\begin{itemize}
\item Parameters: fixed, variables, formulas
\item Materials
\item Model variants
\item Boundary Conditions
\item Post Processing
\end{itemize}
and possibly more ...

\end{frame}

\begin{frame}[fragile]
  \frametitle{Model Properties: JSON}
  From \url{json.org}: ``json is lightweight data-interchange format. It
  is easy for humans to read and write. It is easy for machines to
  parse and generate. It is based on a subset of the JavaScript''

  \inputminted{json}{Codes/prudhomme/fud4/turek.feelpp}
  \begin{itemize}
  \item Use \mintinline{c++}{boost::property_tree}
  \end{itemize}
\end{frame}

\begin{frame}[fragile]
  \frametitle{Model Properties: Parameters}
  Parameters allow to define constants, variables and formulas
  \begin{columns}[c]
    \column{.5\linewidth}
    \begin{minted}[fontsize=\scriptsize,autogobble]{json}
      "Parameters": {
        "Um":{
           "type":"variable",
           "value":0.3,
           "min":1e-4,
           "max":10
         },
         "H": {
           "type":"constant",
           "value":0.41
         }}
    \end{minted}
    \column{.5\linewidth}
    \begin{itemize}
    \item can be used in boundary conditions, material properties or
      in formulas
    \item we should be able to define different kind of parameters and
      customize them (e.g. random variables...)
    \end{itemize}
  \end{columns}
\end{frame}

\begin{frame}[fragile]
  \frametitle{Model Properties: Materials}
  Materials allow to define area with specific properties
  \begin{columns}[c]
    \column{.5\linewidth}
    \begin{minted}{json}
  "Materials":{
    "fluid":{
      // density
      "rho": "1",
      // dynamic_viscosity
      "mu": "0.001"
    }},
\end{minted}
    \column{.5\linewidth}
    \begin{itemize}
    \item Currently support coefficients required by fluid and thermal
      flows with default values
    \item the name of the material is associated directly to a
      physical region in Gmsh
    \end{itemize}
  \end{columns}
\end{frame}

\begin{frame}[fragile]
  \frametitle{Model Properties: Boundary Conditions}
  See \texttt{feel/feelpde/boundaryconditions.hpp}
\begin{columns}[c]
    \column{.5\linewidth}
  \begin{minted}[autogobble,fontsize=\scriptsize]{c++}
    "BoundaryConditions":{
      "velocity": {
        "Dirichlet": {
          "inflow": {
            "expr":"{4*Um*y*( H-y )/H^2,0}:y:Um:H"
          },
          "wall": {
            "expr":"{0,0}"
          },
          "cylinder":{
            "expr":"{0,0}"
          } },
        "Neumann": {
          "outlet": {
            "expr":"{0,0,0,0}"
          }
        }}}, // BoundaryConditions
  \end{minted}
\column{.5\linewidth}
    \begin{itemize}
    \item Support 1 or two expressions (Dirichlet, Neumann, Robin)
    \item Support create of scalar, vector or matrix fields
      \begin{minted}[autogobble,fontsize=\scriptsize]{c++}
        // retrieve a list of all vector fields expressions
        //associated to velocity and Dirichlet conditions
        dirichlet_cond = bc.getVectorFields<dim>
                ( "velocity", "Dirichlet" );

        // retrieve a list of all matrix fields expressions
        // associated to velocity and neumann conditions
        neumann_cond = bc.getMatrixFields<dim>
                ( "velocity", "Neumann" );
      \end{minted}
    \end{itemize}
  \end{columns}
\end{frame}

\begin{frame}[fragile]
  \frametitle{Model Properties: Post Process}
  Define outputs for the model to be computed
\begin{columns}[c]
    \column{.5\linewidth}
  \begin{minted}[autogobble,fontsize=\scriptsize]{c++}
    "PostProcess":
    {
      "Force":["cylinder"],
      "PressureDifference":
      {
        "x1":"{0.15,0.2}",
        "x2":"{0.25,0.2}"
      }}
  \end{minted}
\column{.5\linewidth}
It is up to the model to decide the type of postprocessing it supports
\begin{itemize}
\item Compute drag and lift (Force applied to the structure)
\item Compute fields at points
\item Compute flow rates
\item ...
\end{itemize}
  \end{columns}
\end{frame}

% ================================================================================
\section[CFD]{CFD Toolbox}
%================================================================================

\begin{frame}[standout]{CFD Toolbox}
  \url{http://book.feelpp.org/toolbox/cfd/}
\end{frame}



%\subsection{TurekHron 1}

\begin{frame}[fragile,allowframebreaks]{CFD - TurekHron}

Run  interactive feel++ docker container

\begin{lstlisting}[language=Bash,mathescape=false,emph={docker}]
docker run -it -v $HOME/feel:/feel \
       feelpp/feelpp/feelpp-toolboxes
\end{lstlisting}

You are now in the \textsc{feel++} toolbox docker container.

Place yourself to CFD TurekHron test case directory then execute the
CFD toolbox with given configs

\begin{lstlisting}[language=Bash,mathescape=false, emph={feelpp_toolbox_fluid_2d}]
cd Testcases/CFD/TurekHron/

mpirun -np 4 /usr/local/bin/feelpp_toolbox_fluid_2d \
       --config-file cfd1.cfg
\end{lstlisting}

Take a look to the toolbox config \lstinline{./cfd1.cfg} and model config
\lstinline{cfd1.json}

For more possible options
\begin{lstlisting}[language=Bash,mathescape=false, emph={feelpp_toolbox_fluid_2d}]
feelpp_toolbox_fluid_2d --help
\end{lstlisting}

\end{frame}


% CFD 1

\begin{frame}[fragile,allowframebreaks]{CFD - TurekHron 1 - cfd1.cfg}

Config file \lstinline{~/Testcases/CFD/TurekHron/cfd1.cfg}

\vspace{5mm}
% 35 code lines
\lstinputlisting[language=Bash,mathescape=false]{codes/cfd/turekhron/cfd1.cfg}

\end{frame}


\begin{frame}[fragile,allowframebreaks]{CFD - TurekHron 1 - .json}
Model config file \lstinline{~/Testcases/CFD/TurekHron/cfd1.json}
\vspace{5mm}
% 75 code lines
\lstinputlisting[language=json,mathescape=false]{codes/cfd/turekhron/cfd1.json}

\end{frame}




% CFD 2
%\subsection{TurekHron 2}

\begin{frame}[fragile,allowframebreaks]{CFD - TurekHron 2 - cfd2.cfg}

Config file \lstinline{~/Testcases/CFD/TurekHron/cfd2.cfg}

\vspace{5mm}
% 42 code lines
\lstinputlisting[language=Bash,mathescape=false]{codes/cfd/turekhron/cfd2.cfg}

\end{frame}


\begin{frame}[fragile,allowframebreaks]{CFD - TurekHron 2 - .json}
Model config file \lstinline{~/Testcases/CFD/TurekHron/cfd2.json}
\vspace{5mm}
% 75 code lines
\lstinputlisting[language=json,mathescape=false]{codes/cfd/turekhron/cfd2.json}
\end{frame}





%\subsection{TurekHron 3}
% CFD 3

\begin{frame}[fragile,allowframebreaks]{CFD - TurekHron 3 - cfd3.cfg}

Config file \lstinline{~/Testcases/CFD/TurekHron/cfd3.cfg}

\vspace{5mm}
% 42 code lines
\lstinputlisting[language=Bash,mathescape=false]{codes/cfd/turekhron/cfd3.cfg}
\end{frame}



\begin{frame}[fragile,allowframebreaks]{CFD - TurekHron 3 - .json}
Model config file \lstinline{~/Testcases/CFD/TurekHron/cfd3.json}
\vspace{5mm}
% 75 code lines
\lstinputlisting[language=json,mathescape=false]{codes/cfd/turekhron/cfd3.json}
\end{frame}



%--------------------------------------------------------------------------------

%\subsection{Cavity}


\begin{frame}[fragile,allowframebreaks]{CFD - Cavity}

Run  interactive feel++ docker container

\begin{lstlisting}[language=Bash,mathescape=false,emph={docker}]
docker run -it -v $HOME/feel:/feel \
       feelpp/feelpp/feelpp-toolboxes
\end{lstlisting}

You are now in the \textsc{feel++} toolbox docker container.

Place yourself to CFD Lid-driven-cavity test case directory then execute the
CFD toolbox with given configs

\begin{lstlisting}[language=Bash,mathescape=false, emph={feelpp_toolbox_fluid_2d}]
cd Testcases/CFD/Lid-driven-cavity/

mpirun -np 4 /usr/local/bin/feelpp_toolbox_fluid_2d \
       --config-file cavity2d.cfg
\end{lstlisting}

Take a look to the toolbox config \lstinline{./cavity2d.cfg} and model config
\lstinline{cavity2d.json}

For more possible options
\begin{lstlisting}[language=Bash,mathescape=false, emph={feelpp_toolbox_fluid_2d}]
feelpp_toolbox_fluid_2d --help
\end{lstlisting}

You can test also the 3d case.

\end{frame}



%\begin{frame}[fragile,allowframebreaks]{CFD - TurekHron - cfd1.cfg}
%
%Config file \lstinline{~/Testcases/CFD/TurekHron/cfd1.cfg}
%
%\vspace{5mm}
%% 42 code lines
%\lstinputlisting[language=Bash,mathescape=false,
%firstline=1,lastline=15]
%{codes/cfd/turekhron/cfd1.cfg}
%
%\end{frame}
%
%
%
%\begin{frame}[fragile,allowframebreaks]{CFD - TurekHron - .json}
%Model config file \lstinline{~/Testcases/CFD/TurekHron/cfd1.json}
%\vspace{5mm}
%% 75 code lines
%\lstinputlisting[language=json,mathescape=false,
%firstline=1,lastline=12]
%{codes/cfd/turekhron/cfd1.json}

%\end{frame}









%================================================================================
\section[CSM]{Computation Solid Mechanics toolbox}
%================================================================================

\begin{frame}[standout]{CSM Toolbox}
  \url{http://book.feelpp.org/toolbox/csm/}
\end{frame}

%\subsection{TurekHron}

\begin{frame}[fragile,allowframebreaks]{CSM - TurekHron}

Run  interactive feel++ docker container

\begin{lstlisting}[language=Bash,mathescape=false,emph={docker}]
docker run -it -v $HOME/feel:/feel \
       feelpp/feelpp/feelpp-toolboxes
\end{lstlisting}

You are now in the \textsc{feel++} toolbox docker container.

Place yourself to CSM TurekHron test case directory then execute the
solid toolbox with given configs

\begin{lstlisting}[language=Bash,mathescape=false, emph={feelpp_toolbox_solid_2d}]
cd Testcases/CSM/TurekHron/

mpirun -np 4 /usr/local/bin/feelpp_toolbox_solid_2d \
       --config-file csm3.cfg
\end{lstlisting}

Take a look to the toolbox config \lstinline{./csm3.cfg} and model config
\lstinline{csm3.json}

For more possible options
\begin{lstlisting}[language=Bash,mathescape=false, emph={feelpp_toolbox_solid_2d}]
feelpp_toolbox_solid_2d --help
\end{lstlisting}

\end{frame}




\begin{frame}[fragile,allowframebreaks]{CSM - TurekHron - csm3.cfg}

Config file \lstinline{~/Testcases/solid/TurekHron/csm3.cfg}

\vspace{5mm}
% 42 code lines
\lstinputlisting[language=Bash,mathescape=false]{codes/csm/turekhron2d/csm3.cfg}

\end{frame}





\begin{frame}[fragile,allowframebreaks]{CSM - TurekHron - .json}
Model config file \lstinline{~/Testcases/solid/TurekHron/csm3.json}
\vspace{5mm}
% 75 code lines
\lstinputlisting[language=json,mathescape=false]{codes/csm/turekhron2d/csm3.json}

\end{frame}



%--------------------------------------------------------------------------------

%\subsection{TorsionBar}

\begin{frame}[fragile, allowframebreaks]{CSM - TorsionBar}
Run  interactive feel++ docker container

\begin{lstlisting}[language=Bash,mathescape=false,emph={docker}]
docker run -it -v $HOME/feel:/feel \
       feelpp/feelpp/feelpp-toolboxes
\end{lstlisting}

You are now in the \textsc{feel++} toolbox docker container.

Place yourself to CSM TurekHron test case directory then execute the
solid toolbox with given configs

\begin{lstlisting}[language=Bash,mathescape=false, emph={feelpp_toolbox_solid_2d}]
cd Testcases/CSM/torsionbar

mpirun -np 4 /usr/local/bin/feelpp_toolbox_stress_2d \
       --config-file torsionbar.cfg
\end{lstlisting}

Take a look to the toolbox config \lstinline{./torsionbar.cfg} and model config
\lstinline{torsionbar.json}

For more possible options
\begin{lstlisting}[language=Bash,mathescape=false, emph={feelpp_toolbox_stress_2d}]
feelpp_toolbox_stress_2d --help
\end{lstlisting}

\end{frame}

\begin{frame}[fragile, allowframebreaks]{CSM - TorsionBar - .cfg}
    Config file \lstinline{~/Testcases/CSM/torsionbar/torsionbar.cfg}
\vspace{5mm}
    \lstinputlisting[language=json,mathescape=false]{codes/csm/torsionbar2d/torsionbar.cfg}
\end{frame}

\begin{frame}[fragile, allowframebreaks]{CSM - TorsionBar - .json}
    Model config file \lstinline{~/Testcases/CSM/torsionbar/torsionbar.json}
\vspace{5mm}
\lstinputlisting[language=json,mathescape=false]{codes/csm/torsionbar2d/torsionbar.json}
\end{frame}


%--------------------------------------------------------------------------------

%\subsection{Solenoid}

\begin{frame}[fragile, allowframebreaks]{CSM - Solenoid}
Run  interactive feel++ docker container

\begin{lstlisting}[language=Bash,mathescape=false,emph={docker}]
docker run -it -v $HOME/feel:/feel \
       feelpp/feelpp/feelpp-toolboxes
\end{lstlisting}

You are now in the \textsc{feel++} toolbox docker container.

Place yourself to CSM Solenoid test case directory then execute the
solid toolbox with given configs

\begin{lstlisting}[language=Bash,mathescape=false, emph={feelpp_toolbox_solenoid_3d}]
cd Testcases/CSM/Solenoid

mpirun -np 4 /usr/local/bin/feelpp_toolbox_solenoid_3d \
       --config-file sol.cfg
\end{lstlisting}

Take a look to the toolbox config \lstinline{./sol.cfg} and model config
\lstinline{sol.json}

For more possible options
\begin{lstlisting}[language=Bash,mathescape=false, emph={feelpp_toolbox_solenoid_3d}]
feelpp_toolbox_solenoid_3d --help
\end{lstlisting}

\end{frame}



\begin{frame}[fragile, allowframebreaks]{CSM - Solenoid - .cfg}
Config file \lstinline{~/Testcases/CSM/Solenoid/sol.cfg}
\vspace{5mm}
\lstinputlisting[language=json,mathescape=false]
{codes/csm/solenoid/sol.cfg}
\framebreak
\end{frame}



\begin{frame}[fragile, allowframebreaks]{CSM - Solenoid - .json}
Model config file \lstinline{~/Testcases/CSM/Solenoid/sol.json}
\vspace{5mm}
    \lstinputlisting[language=json,mathescape=false]
{codes/csm/solenoid/sol.json}
\framebreak
\end{frame}





%--------------------------------------------------------------------------------

%\subsection{NAFEMS-LE1}

\begin{frame}[fragile, allowframebreaks]{CSM - NAFEMS-LE1}
Run  interactive feel++ docker container

\begin{lstlisting}[language=Bash,mathescape=false,emph={docker}]
docker run -it -v $HOME/feel:/feel \
       feelpp/feelpp/feelpp-toolboxes
\end{lstlisting}

You are now in the \textsc{feel++} toolbox docker container.

Place yourself to CSM NAFEMS-LE1 test case directory then execute the
solid toolbox with given configs

\begin{lstlisting}[language=Bash,mathescape=false, emph={feelpp_toolbox_stress_2d}]
cd Testcases/CSM/NAFEMS-LE1

mpirun -np 4 /usr/local/bin/feelpp_toolbox_stress_2d \
       --config-file le1.cfg
\end{lstlisting}

Take a look to the toolbox config \lstinline{./le1.cfg} and model config
\lstinline{le1.json}

For more possible options
\begin{lstlisting}[language=Bash,mathescape=false, emph={feelpp_toolbox_stress_2d}]
feelpp_toolbox_stress_2d --help
\end{lstlisting}

\end{frame}

\begin{frame}[fragile, allowframebreaks]{CSM - NAFEMS-LE1 - .cfg}
    Config file \lstinline{~/Testcases/CSM/NAFEMS-LE1/le1.cfg}
\vspace{5mm}
    \lstinputlisting[language=json,mathescape=false]{codes/csm/nafemsle1/le1.cfg}
\framebreak
\end{frame}


\begin{frame}[fragile, allowframebreaks]{CSM - NAFEMS-LE1 - .json}
   Model config file \lstinline{~/Testcases/CSM/NAFEMS-LE1/le1.json}
    \lstinputlisting[language=json,mathescape=false]{codes/csm/nafemsle1/le1.json}
\framebreak
\end{frame}




%--------------------------------------------------------------------------------

%\subsection{NAFEMS-LE10}

\begin{frame}[fragile, allowframebreaks]{CSM - NAFEMS-LE10}
Run  interactive feel++ docker container

\begin{lstlisting}[language=Bash,mathescape=false,emph={docker}]
docker run -it -v $HOME/feel:/feel \
       feelpp/feelpp/feelpp-toolboxes
\end{lstlisting}

You are now in the \textsc{feel++} toolbox docker container.

Place yourself to CSM NAFEMS-LE10 test case directory then execute the
solid toolbox with given configs

\begin{lstlisting}[language=Bash,mathescape=false, emph={feelpp_toolbox_stress_2d}]
cd Testcases/CSM/NAFEMS-LE10

mpirun -np 4 /usr/local/bin/feelpp_toolbox_stress_2d \
       --config-file le10.cfg
\end{lstlisting}

Take a look to the toolbox config \lstinline{./le10.cfg} and model config
\lstinline{le10.json}

For more possible options
\begin{lstlisting}[language=Bash,mathescape=false, emph={feelpp_toolbox_stress_2d}]
feelpp_toolbox_stress_2d --help
\end{lstlisting}

\end{frame}



\begin{frame}[fragile, allowframebreaks]{CSM - NAFEMS-LE10 - .cfg}
Config file \lstinline{~/Testcases/CSM/NAFEMS-LE10/le10.cfg}
\lstinputlisting[language=json,mathescape=false]{codes/csm/nafemsle10/le10.cfg}
\framebreak
\end{frame}


\begin{frame}[fragile, allowframebreaks]{CSM - NAFEMS-LE10 - .json}
Model config file \lstinline{~/Testcases/CSM/NAFEMS-LE10/le10.json}
\lstinputlisting[language=json,mathescape=false]{codes/csm/nafemsle10/le10.json}
\framebreak
\end{frame}


















%%================================================================================
\section[FSI]{Fluid Structure Interaction toolbox}
%================================================================================

\begin{frame}[standout]{FSIB Toolbox}
  \url{http://book.feelpp.org/toolbox/fsi/}
\end{frame}

%\subsection{TurekHron}


\begin{frame}[fragile,allowframebreaks]{FSI - TurekHron}

Run  interactive feel++ docker container

\begin{lstlisting}[language=Bash,mathescape=false,emph={docker}]
docker run -it -v $HOME/feel:/feel \
       feelpp/feelpp/feelpp-toolboxes
\end{lstlisting}

You are now in the \textsc{feel++} toolbox docker container.

Place yourself to FSI TurekHron test case directory then execute the
FSI toolbox with given configs

\begin{lstlisting}[language=Bash,mathescape=false, emph={feelpp_toolbox_fsi_2d}]
cd Testcases/FSI/TurekHron/

mpirun -np 4 /usr/local/bin/feelpp_toolbox_fsi_2d \
       --config-file fsi3.cfg
\end{lstlisting}

Take a look to the toolbox config \lstinline{./fsi3.cfg} and model config
\lstinline{fsi3_solid.json} and \lstinline{fsi3_fluid.json}

For more possible options
\begin{lstlisting}[language=Bash,mathescape=false, emph={feelpp_toolbox_fsi_2d}]
feelpp_toolbox_fsi_2d --help
\end{lstlisting}

\end{frame}



\begin{frame}[fragile,allowframebreaks]{FSI - TurekHron - fsi3.cfg}

Config file \lstinline{~/Testcases/FSI/TurekHron/fsi3.cfg}

\vspace{5mm}
% 42 code lines
\lstinputlisting[language=Bash,mathescape=false]{codes/fsi/turekhron/fsi3.cfg}

\end{frame}


\begin{frame}[fragile,allowframebreaks]{FSI - TurekHron - .json}
Model config file \lstinline{~/Testcases/FSI/TurekHron/fsi3_solid.json}
\vspace{5mm}
% 75 code lines
\lstinputlisting[language=json,mathescape=false]{codes/fsi/turekhron/fsi3_solid.json}

\end{frame}


\begin{frame}[fragile,allowframebreaks]{FSI - TurekHron - .json}
Model config file \lstinline{~/Testcases/FSI/TurekHron/fsi3_fluid.json}
\vspace{5mm}
% 75 code lines
\lstinputlisting[language=json,mathescape=false]{codes/fsi/turekhron/fsi3_fluid.json}

\end{frame}




%--------------------------------------------------------------------------------

%\subsection{Wavepressure 2D}


\begin{frame}[fragile,allowframebreaks]{FSI - WavePressure 2D}

Run  interactive feel++ docker container

\begin{lstlisting}[language=Bash,mathescape=false,emph={docker}]
docker run -it -v $HOME/feel:/feel \
       feelpp/feelpp/feelpp-toolboxes
\end{lstlisting}

You are now in the \textsc{feel++} toolbox docker container.

Place yourself to FSI WavePressure test case directory then execute the
FSI toolbox with given configs

\begin{lstlisting}[language=Bash,mathescape=false, emph={feelpp_toolbox_fsi_2d}]
cd Testcases/FSI/wavepressure2d/

mpirun -np 4 /usr/local/bin/feelpp_toolbox_fsi_2d \
       --config-file fsi3.cfg
\end{lstlisting}

Take a look to the toolbox config \lstinline{./wavepressure2d.cfg} and model config
\lstinline{wavepressure2d_solid.json} and \lstinline{wavepressure2d_fluid.json}

For more possible options
\begin{lstlisting}[language=Bash,mathescape=false, emph={feelpp_toolbox_fsi_2d}]
feelpp_toolbox_fsi_2d --help
\end{lstlisting}

\end{frame}


\begin{frame}[fragile,allowframebreaks]{FSI - WavePressure 2D - .cfg}

Config file \lstinline{~/Testcases/FSI/wavepressure2d/wavepressure2d.cfg}

\vspace{5mm}
Config file \lstinline{~/Testcases/FSI/wavepressure2d/wavepressure2d.json}
% 42 code lines
\lstinputlisting[language=Bash,mathescape=false]{codes/fsi/wavepressure2d/wavepressure2d.cfg}

\end{frame}


\begin{frame}[fragile,allowframebreaks]{FSI - WavePressure 2D - .json}
Model config file \lstinline{~/Testcases/FSI/wavepressure2d/wavepressure2d_solid.json}
\vspace{5mm}
% 75 code lines
\lstinputlisting[language=json,mathescape=false]
{codes/fsi/wavepressure2d/wavepressure2d_solid.json}
\end{frame}



\begin{frame}[fragile,allowframebreaks]{FSI - WavePressure 2D - .json}
Model config file \lstinline{~/Testcases/FSI/wavepressure2d/wavepressure2d_fluid.json}
\vspace{5mm}
% 75 code lines
\lstinputlisting[language=json,mathescape=false]
{codes/fsi/wavepressure2d/wavepressure2d_fluid.json}
\end{frame}





%--------------------------------------------------------------------------------


%\subsection{Wavepressure 3D}


\begin{frame}[fragile,allowframebreaks]{FSI - WavePressure 3D}

Run  interactive feel++ docker container

\begin{lstlisting}[language=Bash,mathescape=false,emph={docker}]
docker run -it -v $HOME/feel:/feel \
       feelpp/feelpp/feelpp-toolboxes
\end{lstlisting}

You are now in the \textsc{feel++} toolbox docker container.

Place yourself to FSI WavePressure test case directory then execute the
FSI toolbox with given configs

\begin{lstlisting}[language=Bash,mathescape=false, emph={feelpp_toolbox_fsi_2d}]
cd Testcases/FSI/wavepressure3d/

mpirun -np 4 /usr/local/bin/feelpp_toolbox_fsi_2d \
       --config-file fsi3.cfg
\end{lstlisting}

Take a look to the toolbox config \lstinline{./wavepressure3d.cfg} and model config
\lstinline{wavepressure3d_solid.json} and \lstinline{wavepressure3d_fluid.json}

For more possible options
\begin{lstlisting}[language=Bash,mathescape=false, emph={feelpp_toolbox_fsi_2d}]
feelpp_toolbox_fsi_2d --help
\end{lstlisting}

\end{frame}


\begin{frame}[fragile,allowframebreaks]{FSI - WavePressure 3D - .cfg}

Config file \lstinline{~/Testcases/FSI/wavepressure3d/wavepressure3d.cfg}

\vspace{5mm}
% 42 code lines
\lstinputlisting[language=Bash,mathescape=false]{codes/fsi/wavepressure3d/wavepressure3d.cfg}

\end{frame}


\begin{frame}[fragile,allowframebreaks]{FSI - WavePressure 3D - .json}
Model config file \lstinline{~/Testcases/FSI/wavepressure3d/wavepressure3d_solid.json}
\vspace{5mm}
% 75 code lines
\lstinputlisting[language=json,mathescape=false]{codes/fsi/wavepressure3d/wavepressure3d_solid.json}
\end{frame}


\begin{frame}[fragile,allowframebreaks]{FSI - WavePressure 3D - .json}
Model config file \lstinline{~/Testcases/FSI/wavepressure3d/wavepressure3d_fluid.json}
\vspace{5mm}
% 75 code lines
\lstinputlisting[language=json,mathescape=false]{codes/fsi/wavepressure3d/wavepressure3d_fluid.json}
\end{frame}



%================================================================================
\section[HeatTransfer]{Heat Transfer toolbox}
%================================================================================

\begin{frame}[standout]{Heat Transfer Toolbox}
  \url{http://book.feelpp.org/toolbox/heat/}
\end{frame}

%\begin{frame}[fragile]{Building}
%\end{frame}



%\subsection{Thermo 2D}


\begin{frame}[fragile,allowframebreaks]{HeatTransfer - Thermo 2D}

Run an interactive feel++ docker container

\begin{lstlisting}[language=Bash,mathescape=false,emph={docker}]
docker run -it -v $HOME/feel:/feel \
       feelpp/feelpp/feelpp-toolboxes
\end{lstlisting}

You are now in the \textsc{feel++} toolbox docker container.

Place yourself to HeatTransfer Thermo 2d test case directory then execute the
HeatTransfer toolbox with given configs

\begin{lstlisting}[language=Bash,mathescape=false, emph={feelpp_toolbox_thermodyn_2d}]
cd Testcases/HeatTransfer/thermo2d/

mpirun -np 4 /usr/local/bin/feelpp_toolbox_thermodyn_2d \
       --config-file fsi3.cfg
\end{lstlisting}

Take a look to the toolbox config \lstinline{./thermo2d.cfg} and model config
\lstinline{thermo2d.json}.

For more possible options
\begin{lstlisting}[language=Bash,mathescape=false, emph={feelpp_toolbox_fsi_2d}]
feelpp_toolbox_thermodyn_2d --help
\end{lstlisting}

\end{frame}


\begin{frame}[fragile,allowframebreaks]{HeatTransfer - Thermo 2D - .cfg}

Config file \lstinline{~/Testcases/HeatTransfer/thermo2d/thermo2d.cfg}

\vspace{5mm}
% 42 code lines
\lstinputlisting[language=Bash,mathescape=false]{codes/heattransfer/thermo2d/thermo2d.cfg}

\end{frame}


\begin{frame}[fragile,allowframebreaks]{HeatTransfer - Thermo 2D - .json}
Model config file \lstinline{~/Testcases/HeatTransfer/thermo2d/thermo2d.json}
\vspace{5mm}
% 75 code lines
\lstinputlisting[language=json,mathescape=false]{codes/heattransfer/thermo2d/thermo2d.json}
\end{frame}




%================================================================================
\section[ThermoElectric]{Thermo Electric toolbox}
%================================================================================

\begin{frame}[standout]{Thermoelectric Toolbox}
  \url{http://book.feelpp.org/toolbox/thermoelectric/}

\end{frame}

%\subsection{Test 2D}

\begin{frame}[fragile,allowframebreaks]{ThermoElectric - Test 2D}

Run an interactive feel++ docker container

\begin{lstlisting}[language=Bash,mathescape=false,emph={docker}]
docker run -it -v $HOME/feel:/feel \
       feelpp/feelpp/feelpp-toolboxes
\end{lstlisting}

You are now in the \textsc{feel++} toolbox docker container.

Place yourself to ThermoElectric test 2d test case directory then execute the
ThermoElectric toolbox with given configs

\begin{lstlisting}[language=Bash,mathescape=false, emph={feelpp_toolbox_thermoelectric_2d}]
cd Testcases/ThermoElectric/test/

mpirun -np 4 /usr/local/bin/feelpp_toolbox_thermoelectric_2d \
       --config-file test-thermoelectric.cfg
\end{lstlisting}

Take a look to the toolbox config \lstinline{./test-thermoelectric.cfg} and model config
\lstinline{test-thermoelectric.json}.

For more possible options
\begin{lstlisting}[language=Bash,mathescape=false, emph={feelpp_toolbox_thermoelectric_2d}]
feelpp_toolbox_thermoelectric_2d --help
\end{lstlisting}

\end{frame}


\begin{frame}[fragile,allowframebreaks]{ThermoElectric - Test 2D - .cfg}

Config file \lstinline{~/Testcases/ThermoElectric/test/test-thermoelectric.cfg}

\vspace{5mm}
% 42 code lines
\lstinputlisting[language=Bash,mathescape=false]{codes/thermoelectric/test/test-thermoelectric.cfg}

\end{frame}


\begin{frame}[fragile,allowframebreaks]{ThermoElectric - Test 2D - .json}
Model config file \lstinline{~/Testcases/ThermoElectric/test/test-thermoelectric.json}
\vspace{5mm}
% 75 code lines
\lstinputlisting[language=json,mathescape=false]{codes/thermoelectric/test/test-thermoelectric.json}
\end{frame}

\section{AngioTK}
\subsection{Using Feel++ with AngioTk}
\begin{frame}{Using Feel++}

AngioTk is a platform for mesh reconstruction and Feel++ is one of its
components:
\begin{itemize}
\item Files generated by volumic step can be used directly with Feel++
\item AngioTK generates automatically markers for boundary conditions
  randomly, it needs to be updated eventually to get more meaningful
  markers, see the \texttt{.desc} file
\item AngioTk generates a volumic marker that can be changed by hand
  if needed to have a more meaningful marker (to be changed in
  \texttt{.msh} file)
\end{itemize}

\end{frame}
\subsection{Spring}

\begin{frame}[fragile]{AngioTK Spring}

Run  interactive feel++ docker container

\begin{lstlisting}[language=Bash,mathescape=false,emph={docker}]
docker run --rm -it -v $HOME/feel:/feel \
       feelpp/angiotk:master-ubuntu-16.10
\end{lstlisting}

You are now in the \textsc{Feel++} toolbox docker container.

Place yourself to AngioTK Spring test case directory

\begin{lstlisting}[language=Bash,mathescape=false]
cd Testcases/angiotk/MeshFromCenterlines/spring
\end{lstlisting}

and follow the instructions on Github

\centerline{\tiny\url{https://github.com/vivabrain/angiotk/tree/master/Examples/MeshFromCenterlines/spring}}

\end{frame}


%%% Local Variables:
%%% mode: latex
%%% TeX-master: "lecture.feelpp.slides"
%%% End:
