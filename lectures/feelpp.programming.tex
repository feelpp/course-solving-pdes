\section{Programming with Feel++}


\section{Introduction to Feel++}
\begin{frame}{General Presentation}

\end{frame}

\section{Installation with Docker}
\begin{frame}

\end{frame}
%%% Local Variables:
%%% mode: latex
%%% TeX-master: "lecture.feelpp.slides"
%%% End:


\end{document}

\begin{frame}{Programming is easy}

\end{frame}


\section{Programming and Execution Environment}


\begin{frame}{Programming environment}
  \begin{itemize}
  \item Use CDash
    \centerline{\href{http://my.cdash.org/index.php?project=Feel\%2B\%2B&display=project}{CDash
        Feel++ Web Page}}
  \item  Use Travis (see \texttt{.travis.yml}) for Feel++ as well as
    private projects
    \centerline{\href{https://travis-ci.org/feelpp/feelpp}{Travis
        Feel++ Web Page}}
  \item Use DDT/Map/Performance Reports
    \centerline{\url{https://github.com/feelpp/feelpp/wiki/ddt-map}}
  \end{itemize}
\end{frame}

\begin{frame}[fragile]{Programming environment:Embracing C++11 standard }
    \mintinline{c++}{auto} keyword and its dual
    \mintinline{c++}{decltype}
    \begin{cppcode}
      auto mesh = loadMesh(_mesh=new Mesh<Simplex<2>>);
      auto Xh = Pch<3>( mesh );
      auto u = Xh->element();
      auto a = form2(_test=Xh,_trial=Xh);
      auto l = form1(_test=Xh);
      auto e = exporter(_mesh=mesh);
      Pch_ptrtype<decltype(mesh),2> Vh=Pch<2>(mesh);
    \end{cppcode}
\end{frame}

\begin{frame}[fragile]{Programming environment:Embracing C++11 standard }

    \mintinline{c++}{lambda} functions

      \begin{minted}[]{c++}
        auto f = [=]( element_type& p ){
          p.on(_range=elements(mesh),_expr=cst(1.)); }
        auto u = Xh->element();
        f( u );
      \end{minted}
\end{frame}

\begin{frame}[fragile]{Programming environment:Embracing C++11 standard}
    Constructor delegation (see \mintinline{c++}{Environment} class)

    \begin{minted}[]{c++}
      class A {
        A(): A(0) {}
        A( int j ): i(j) {}
        int i;
      };
    \end{minted}
\end{frame}

\begin{frame}[fragile]{Programming environment:Embracing C++11 standard }
  rvalue, universal references, \mintinline{c++}{std::move},
  \mintinline{c++}{std::forward}
  \begin{minted}[autogobble,fontsize=\small]{c++}
    template<typename SpaceType, typename ExprType,
             typename... Args>
boost::shared_ptr<NullSpace<double>>
nullspace_ptr( boost::shared_ptr<SpaceType> Xh,
               ExprType&& e, Args... args )
{
    auto K = boost::make_shared<NullSpace<double>>();
    nullspace( *K, Xh, e, args...);
    return K;
}
  \end{minted}
\end{frame}

\begin{frame}[fragile]{Programming environment:Embracing C++11 standard }
     Using \mintinline{c++}{using} for aliases
\begin{minted}[autogobble,fontsize=\small]{c++}
template<typename MeshT>
using mesh_type = typename
   mpl::if_<Feel::detail::is_shared_ptr<MeshT>,
            mpl::identity<typename MeshT::element_type>,
            mpl::identity<MeshT>>::type::type;
auto mesh = loadMesh(_mesh=new Mesh<Simplex<2>>);
CHECK(boost::is_same<mesh_type<delctype(mesh)>,
                     Mesh<Simplex<2>>>::value==true);
auto mesh=...
Pch_ptrtype<decltype(mesh),2> Xh = Pch<2>(mesh);
\end{minted}
\end{frame}

\begin{frame}[fragile]
  \frametitle{Testsuite: almost there}
%  \mint{shell}{cd testsuite && ctest -j10 -R .}
  \begin{minted}[fontsize=\tiny]{shell}
    97% tests passed, 7 tests failed out of 217

Label Time Summary:
testalg              =  30.09 sec
testcore             =  22.03 sec
testcrb              =  23.26 sec
testdiscr            =  91.51 sec
testfilters          =  15.00 sec
testintegration      =  21.96 sec
testinterpolation    =  13.33 sec
testleaks            =  22.74 sec
testmaterial         =   0.88 sec
testmath             =   1.14 sec
testmesh             =  22.60 sec
testmodels           =   3.92 sec
testopt              =   1.42 sec
testpoly             =  35.05 sec
testts               =   4.93 sec
testvf               =  11.71 sec

Total Test time (real) =  89.83 sec

The following tests FAILED:
	 37 - feelpp_test_eigenmode-np-6 (Timeout)
	 38 - feelpp_test_eigenmode-np-1 (Timeout)
	 77 - feelpp_test_modelproperties-np-6 (Failed)
	 78 - feelpp_test_modelproperties-np-1 (Failed)
	 79 - feelpp_test_form_interpolation-np-1 (Timeout)
	157 - feelpp_test_project-np-1 (Failed)
	170 - feelpp_test_mortar-np-1 (Failed)
  \end{minted}
\end{frame}

\begin{frame}{Environment}
  \begin{itemize}
  \item Multi-threading support
    \begin{itemize}
    \item \mintinline{c++}{single} only one thread
    \item \mintinline{c++}{funneled} only main thread make MPI calls
    \item \mintinline{c++}{serialized} Only one thread at the time do
      MPI calls
    \item \mintinline{c++}{multiple} Multiple thread may do MPI calls.
    \end{itemize}
  \item Internal reorganisation that uses c++11 construction delegation
  \end{itemize}

\end{frame}
\begin{frame}
  \frametitle{Environment}
  \inputminted[fontsize=\scriptsize]{c++}{Codes/prudhomme/fud4/env.cpp}
\end{frame}

\begin{frame}
  \frametitle{Miscellaneous/Contrib}
  \begin{itemize}
  \item \texttt{Paralution} : solvers/preconditioners for hybrid computing
  \item \texttt{NT2} : Matlab in C++
  \item \texttt{NLOpt} : A collection of algorithms for Non-Linear optimization
  \item \texttt{IpOpt}: Interior point method
  \end{itemize}
\end{frame}

\section{Mesh }

\begin{frame}[fragile]{Mesh}
  \begin{itemize}
  \item filters can take a list of markers and most
    algorithms(integration,interpolation) have now been modified to
    support a list of filters
  \item when \mintinline{c++}{markedpoints},
    \mintinline{c++}{markededges} are non-empty then we attach now
    the list of elements (which are not ghost) to which these entities
    are associated
  \item Memory : we do not store anymore a lot of data but recompute
    them as needed (see \verb+feature/edge+)
  \end{itemize}
\end{frame}

\section{Finite Element Framework}

\begin{frame}[fragile]{Vectorial Finite Elements}
  $H_{\mathrm{hdiv}}$ and $H_{\mathrm{curl}}$ finite elements have
  entered Feel++ and are now used daily (see \verb+@cdaversin+,
  \verb+romainhild+, \verb+vhuber+
  presentation)
  \begin{itemize}
  \item Raviart-Thomas elements
  \item First kind Nedelec elements
  \end{itemize}
\end{frame}

\begin{frame}[fragile]
  \frametitle{Raviart-Thomas}
  \begin{itemize}
  \item Support only lowest order $\mathbb{R}\mathbb{T}_0$ (higher order to be debugged)
  \item BDM (full polynomial space) should be supported
    (required Nedelec 1st kind)
  \end{itemize}
  \inputminted[fontsize=\scriptsize]{c++}{Codes/prudhomme/fud4/rt.cpp}
\end{frame}

\begin{frame}[fragile]
  \frametitle{Nedelec}
  \begin{itemize}
  \item Support only  lowest order first kind $\mathbb{N}_0$ (higher order to be
    debugged)
  \item Second kind (full polynomial space) should be supported
    (require RT)
  \end{itemize}
  \inputminted[fontsize=\scriptsize]{c++}{Codes/prudhomme/fud4/ned.cpp}
\end{frame}

\begin{frame}[fragile]
  \frametitle{Interpolation}
  \begin{itemize}
  \item We have a general interpolation framework now: we use to
    support only Lagrange finite elements.
  \item we can now use the interpolant associated to the supported
    nodal finite elements space
  \end{itemize}
  \inputminted[fontsize=\scriptsize]{c++}{Codes/prudhomme/fud4/interp.cpp}
\end{frame}
\begin{frame}{Interpolation: Local Interpolant}
  \begin{block}{Local Interpolant}
    Consider the finite element $(K\subset\mathbb{R}^d,P=\spn\{\phi_i\},\Sigma=\{\ell_i,
    i=1\ldots \dim P\})$, the local interpolant reads
    \begin{equation}
      \pi_h u = \sum_i \ell_i(u) \phi_i \ \in P
    \end{equation}
    The local interpolant is now implemented in the associated finite element
    class. It is generic from the global interpolant point of view
  \end{block}
\end{frame}
\begin{frame}{Interpolation: Global Interpolant}
  \begin{block}{Global Interpolant}
    The global interpolant requires
    \begin{itemize}
    \item the reference finite element
    $(\hat{K},\hat{P},\hat{\Sigma})$
     \item the geometric
       transformation $\psi_K : \hat{K} \rightarrow K$ and possibly
       associated Piola transforms (e.g. Raviart-Thomas and Nedelec).
     \item for Raviart-Thomas and Nedelec a unique definition of normal and edge
       tangents respectively (use permutations developed for high order)
    \end{itemize}
    \begin{alertblock}{}
      \alert{Pass only the function $u$ to be interpolated (and not \emph{e.g.} $u
      \cdot n$ for Raviart-Thomas)!}
    \end{alertblock}
  \end{block}
\end{frame}

\section{Time Stepping}

\begin{frame}
  \frametitle{Time Stepping}

  A nice interface that handles I/O and restarts

  \begin{itemize}
  \item BDF$_q,\ q=1\ldots 4$
  \item Newmark
  \item On going development: an adaptative time stepping strategy
    \texttt{TR-AB2} (Implicit TR, Explit Adams Bashforth of order 2)
  \end{itemize}

  \alert{Issue:} parallel data format for time stepping storage
\end{frame}
\section{Backends}

\begin{frame}[fragile]
  \frametitle{Backends}
  \alert{Backends} are the working horse of Feel++ (a part from
  assembly/interpolation).
  \begin{itemize}
  \item create submatrices (e.g. for domain decomposition or multiphysics)
  \item deep integration through cfg files of PETSc
    solver/preconditioners into Feel++
  \item support of multigrid preconditioners : \textsc{ml},
    \textsc{gamg}, \textsc{boomeramg}
  \item support control (33 options) of \textsc{mumps} (see
    \href{https://github.com/feelpp/feelpp/issues/499}{Issue 499 on Github}, for \verb+i=1...33+
    \begin{minted}[autogobble]{shell}
      --<backend prefix>.pc-factor-mumps.icntl-<i>=<value>
      or in .cfg file
      [<backend prefix>]
      pc-factor-mumps.icntl-<i>=<value>
    \end{minted}
  \item monitors are customized with respect to the name of the
    backend (easier now to look at embedded ksp solvers convergence)
  \item
  \end{itemize}
\end{frame}
\begin{frame}[fragile]
  \frametitle{Backends: Null Spaces}
  Feel++ now support the definition of (near) null space and pass that
  information to the backend to ensure scalability (e.g. in
  elasticity) and in particular for multigrid methods

  \inputminted[fontsize=\scriptsize]{c++}{Codes/prudhomme/fud4/nullspace.cpp}
\end{frame}

\begin{frame}[fragile]
  \frametitle{Backends: In the pipeline}
  \begin{itemize}
  \item provide a deep interface to SLEPc similar to PETSc (required by
    NS2++ project see \verb+romainhild+ presentation)
  \item provide an interface to non-linear preconditioners(for
    non-linear solvers), PETSc provides a nice framework for that
  \item Other backends:
    \begin{itemize}
    \item improve support for \alert{Eigen}
    \item start support for \alert{Paralution}
    \item start support for \alert{NT$_2$}
    \item have another look at Trilinos framework?
    \end{itemize}
  \end{itemize}
\end{frame}

\begin{frame}[fragile]
  \frametitle{In-House preconditioners}
  \begin{itemize}
  \item \verb+@prj-+ presents HPDDM
  \item \verb+asamake+ presents the substructuring preconditioner for
    the h-p mortar
  \end{itemize}
\end{frame}

\section{Operator Framework}
\begin{frame}
  \frametitle{Operator Framework: why ?}
  We support PETSc \texttt{fieldsplit} extensively: it enables
  multi-physics proeconditioners. However we had difficulties getting
  good results for some incompressible Navier-Stokes problems.

  \centering
  \vfill\vspace{1em}\usebeamerfont{section title}{\scshape Introduce Operator Framework}\vfill
\end{frame}

\begin{frame}[fragile]
  \frametitle{Operator Framework: Operator Interface}
  The framework provides a basic operator interface
  \begin{minted}[fontsize=\scriptsize,autogobble]{c++}
    // operator A
    apply(vector_ptrtype x, vector_ptrtype y )
    { /*y = A x;*/ }
    // operator A^-1
    applyInverse(vector_ptrtype x, vector_ptrtype y )
    { /*y = A^-1 x;*/ }
    // use transpose
    setUseTranspose(bool) -> true: A^T, false: A
    // domain and image map: necessary when domain and image space are
    // not of the same dimension
    domainMap()
    imageMap()
    // gives a name or label to the operators and generate names from
    // subsequent operations
    label()
    setLabel(std::string)
  \end{minted}
\end{frame}

\begin{frame}[fragile]
  \frametitle{Operator Framework: Operators}
  Now that the operator interface is define, we define operators
  \begin{itemize}
  \item \mintinline{c++}{op(A)}: operator associated to a matrix
  \item \mintinline{c++}{C=compose(A,B)}: composition of two operators
    $C=A B$
  \item \mintinline{c++}{C=inv(A)}: inverse of operator $C=A^{-1}$
  \item \mintinline{c++}{C=scale(A,alpha)}: scale operator $C=\alpha
    A$
  \item \mintinline{c++}{C=diag(A)}: diagonal of operator $C=\diag A$
  \item \mintinline{c++}{C=diag(A,B)}: diagonal operator $D=
    \begin{pmatrix}
      A & 0\\
      0 & B
    \end{pmatrix}
$
  \item \mintinline{c++}{D=triu(A,B,C)}: upper triangular operator  $D=
    \begin{pmatrix}
      A & B\\
      0 & C
    \end{pmatrix}
$
\item \mintinline{c++}{D=tril(A,B,C)}: lower triangular operator $D=
    \begin{pmatrix}
      A & 0\\
      B & C
    \end{pmatrix}
$
\item actually the previous three operations can handle a larger number of blocks
  \end{itemize}
\end{frame}

\begin{frame}[fragile]
  \frametitle{Operator Framework : application to Stokes/Navier-Stokes...}
  Consider the following system
  \begin{equation}
    \label{eq:1}
    A = \begin{pmatrix}
      F & B^T\\
      B & 0
    \end{pmatrix}
  \end{equation}
  where $F$ is the diffusion or convection-diffusion operator on
  velocity and $B$, $B^T$ the coupling blocks between velocity and
  pressure.

  A preconditioner for
  $A$ reads (see \texttt{@ranine} presentation)
  \begin{equation}
    \label{eq:2}
    P = \begin{pmatrix}
      F & B^T\\
      0 & \hat{S}
    \end{pmatrix}
  \end{equation}
  where $\hat{S}$ is an efficient approximation of the Schur
  complement. One can take for low Reynolds number
  $\hat{S}=-\frac{1}{\nu}M_p$ where $M_p$ is the pressure mass matrix.
\end{frame}
\begin{frame}[fragile]
  \frametitle{Operator Framework : application to Stokes/Navier-Stokes...}
  \inputminted[fontsize=\scriptsize,autogobble,mathescape]{c++}{Codes/prudhomme/fud4/op.cpp}
\end{frame}

\section{DSEL: New Keywords and Features}
\begin{frame}[fragile]{DSEL: New Keywords and Features}
  \begin{itemize}
  \item \mintinline{c++}{rand<double|int>(a,b)} : generate a float or
    discrete random number between \texttt{a} and \texttt{b} depending on the type of \texttt{a}
    and \texttt{b}
  \item \mintinline{c++}{floor(expr),ceil(expr)} : floor and ceil of
    an expression
  \item \mintinline{c++}{atan2} : compute atan2 of expression
  \item \mintinline{c++}{cast<type>(expr)} : cast an expression to
    type \mintinline{c++}{type}.
  \item \mintinline{c++}{%} : modulo operation (integers only,
                         %requires casting)
  \end{itemize}
\end{frame}
\begin{frame}[fragile]
  \frametitle{New Keywords and Features}
  \mintinline{c++}{idf(.)} : enables functors in the language (not
  really new, not much advertised but starts to be used)
  \begin{minted}[fontsize=\scriptsize,autogobble]{c++}
        struct kappa
        {
          static const size_type context = vm::JACOBIAN|vm::POINT;
          typedef double value_type;
          typedef value_type evaluate_type;
          static const uint16_type rank = 0;
          static const uint16_type imorder = 1;
          static const bool imIsPoly = true;
          value_type val = 1;
          double operator()(uint16_type, uint16_type,
                            ublas::vector<double> const& x,
                            ublas::vector<double> const&) const
          { return x[0]+x[1]; }
        };
        std::cout << integrate(_range=elements(mesh),
                               _expr=idf(kappa)).evaluate();
      \end{minted}

\end{frame}
\begin{frame}[fragile]
  \frametitle{New Keywords and Features}
  \mintinline{c++}{val(.)} : store evaluation of expression at
  quadrature points, allows to speedup assembly process of bi/linear
  forms (not new but not well known)
  \begin{minted}[fontsize=\scriptsize,autogobble]{c++}
    auto mesh=...;
    auto Xh=...;
    auto u = Xh->element();
    auto f = sin(pi*Px())*cos(pi*Py())*cos(pi*Pz());
    auto a = form2(_test=Xh,_trial=Xh);
    a = integrate(_range=elements(mesh),
                  _expr=val(f)*idt(u)*id(u)).evaluate();
  \end{minted}
\end{frame}

\begin{frame}[fragile]{DSEL: New Keywords and Features}
  \alert{Support component wise Dirichlet conditions}
  When manipulating vector fields, setting Dirichlet conditions per
  component or working with the components is useful
  \inputminted[fontsize=\scriptsize]{c++}{Codes/prudhomme/fud4/elas.cpp}
\end{frame}

\begin{frame}[fragile]{DSEL: New Keywords and Features}
  \alert{\mintinline{c++}{on} and interpolation now support
  interpolation at points (2D/3D) and edges (3D)}
\begin{columns}[c]
  \column{.5\linewidth}
  \inputminted[fontsize=\tiny]{c++}{Codes/prudhomme/fud4/on.cpp}
  \column{.5\linewidth}
  \includegraphics[width=.8\linewidth]{dirac.png}
\end{columns}

\end{frame}

\section{Model Properties Framework}

\begin{frame}{Model Properties: Why?}
  We have a nice DSEL that allows to write variational formulations,
  we have the cfg files as well as Ginac. However everyone comes up
  with a different solution, we need to have a more unified way to
  describe our models.

  \centering
  \vfill\vspace{1em}\usebeamerfont{section title}{\scshape Introduce
    Model Properties Framework}\vfill
\end{frame}

\begin{frame}{Model Properties: What?}

What are model properties?
\begin{itemize}
\item Parameters: fixed, variables, formulas
\item Materials
\item Model variants
\item Boundary Conditions
\item Post Processing
\end{itemize}
and possibly more ...

\end{frame}

\begin{frame}[fragile]
  \frametitle{Model Properties: JSON}
  From \url{json.org}: ``json is lightweight data-interchange format. It
  is easy for humans to read and write. It is easy for machines to
  parse and generate. It is based on a subset of the JavaScript''

  \inputminted{json}{Codes/prudhomme/fud4/turek.feelpp}
  \begin{itemize}
  \item Use \mintinline{c++}{boost::property_tree}
  \end{itemize}
\end{frame}

\begin{frame}[fragile]
  \frametitle{Model Properties: Parameters}
  Parameters allow to define constants, variables and formulas, see \texttt{feel/feelmodels/modelparameter.hpp}
  \begin{columns}[c]
    \column{.5\linewidth}
    \begin{minted}[fontsize=\scriptsize,autogobble]{json}
      "Parameters": {
        "Um":{
           "type":"variable",
           "value":0.3,
           "min":1e-4,
           "max":10
         },
         "H": {
           "type":"constant",
           "value":0.41
         }}
    \end{minted}
    \column{.5\linewidth}
    \begin{itemize}
    \item can be used in boundary conditions, material properties or
      in formulas
    \item we should be able to define different kind of parameters and
      customize them (e.g. random variables...)
    \end{itemize}
  \end{columns}
\end{frame}

\begin{frame}[fragile]
  \frametitle{Model Properties: Materials}
  Materials allow to define area with specific properties, see \texttt{feel/feelmodels/modelmaterials.hpp}
  \begin{columns}[c]
    \column{.5\linewidth}
    \begin{minted}{json}
  "Materials":{
    "fluid":{
      // density
      "rho": "1",
      // dynamic_viscosity
      "mu": "0.001"
    }},
\end{minted}
    \column{.5\linewidth}
    \begin{itemize}
    \item Currently support coefficients required by fluid and thermal
      flows with default values
    \item the name of the material is associated directly to a
      physical region in Gmsh
    \end{itemize}
  \end{columns}
\end{frame}

\begin{frame}[fragile]
  \frametitle{Model Properties: Boundary Conditions}
  See \texttt{feel/feelpde/boundaryconditions.hpp}
\begin{columns}[c]
    \column{.5\linewidth}
  \begin{minted}[autogobble,fontsize=\scriptsize]{c++}
    "BoundaryConditions":{
      "velocity": {
        "Dirichlet": {
          "inflow": {
            "expr":"{4*Um*y*( H-y )/H^2,0}:y:Um:H"
          },
          "wall": {
            "expr":"{0,0}"
          },
          "cylinder":{
            "expr":"{0,0}"
          } },
        "Neumann": {
          "outlet": {
            "expr":"{0,0,0,0}"
          }
        }}}, // BoundaryConditions
  \end{minted}
\column{.5\linewidth}
    \begin{itemize}
    \item Support 1 or two expressions (Dirichlet, Neumann, Robin)
    \item Support create of scalar, vector or matrix fields
      \begin{minted}[autogobble,fontsize=\scriptsize]{c++}
        // retrieve a list of all vector fields expressions
        //associated to velocity and Dirichlet conditions
        dirichlet_cond = bc.getVectorFields<dim>
                ( "velocity", "Dirichlet" );

        // retrieve a list of all matrix fields expressions
        // associated to velocity and neumann conditions
        neumann_cond = bc.getMatrixFields<dim>
                ( "velocity", "Neumann" );
      \end{minted}
    \end{itemize}
  \end{columns}
\end{frame}

\begin{frame}[fragile]
  \frametitle{Model Properties: Post Process}
  Define outputs for the model to be computed, see \texttt{feel/feelmodels/modelpostprocess.hpp}
\begin{columns}[c]
    \column{.5\linewidth}
  \begin{minted}[autogobble,fontsize=\scriptsize]{c++}
    "PostProcess":
    {
      "Force":["cylinder"],
      "PressureDifference":
      {
        "x1":"{0.15,0.2}",
        "x2":"{0.25,0.2}"
      }}
  \end{minted}
\column{.5\linewidth}
It is up to the model to decide the type of postprocessing it supports
\begin{itemize}
\item Compute drag and lift (Force applied to the structure)
\item Compute fields at points
\item Compute flow rates
\item ...
\end{itemize}
  \end{columns}
\end{frame}

\plain{\small\texttt{feel/feelmodels/fluid/fluidmechanics.hpp}}

\section{Reduced Basis}
\begin{frame}[fragile]
  \frametitle{Reduced Basis}
  \begin{itemize}
  \item A Reduced Basis framework to handle  non-linear multiphysic
    problems, the enablers are
    \begin{itemize}
    \item EIM, Empirical Interpolation Method
    \item RB Reduced Basis method
    \end{itemize}
  \item \verb+@cdaversin+ presents a Simultaneous EIM-RB (SER) algorithm
    that is a huge step forward using RB in industrial context
  \item \verb+@jbwahl+ presents the extension of the framework to
    saddle point problems
  \end{itemize}
\end{frame}

\section{Post Processing}

\begin{frame}[fragile]
  \frametitle{Importers}
  \begin{itemize}
  \item We import data using Gmsh, very good so far
  \item For large scale computation we need now a dataformat that
    scales when (tens of) thousands of cores read the data files
  \item AngioTK with Kitware and the Vivabrain project, see \verb+@vincentchabannes+ presentation for a
    framework from medical to numerical simulations !
  \end{itemize}
\end{frame}
\begin{frame}[fragile]{Exporters}
  \begin{columns}[c]
    \column{.5\linewidth}
    \begin{itemize}
    \item Formats Ensight Gold, HDF5, VTK, Gmsh (see @aancel presentation)
    \item Exporters
      \begin{itemize}
      \item by default consider coordinates are changing, if geometry
        does not change, use \texttt{static}
      \end{itemize}
    \end{itemize}
    \column{.5\linewidth}
    \inputminted[fontsize=\tiny]{c++}{Codes/prudhomme/fud4/exporter.cpp}
    %\inputminted{text}{exporter.cfg}
  \end{columns}

\end{frame}

\begin{frame}[fragile]{In-Situ Visualisation}
  \begin{columns}[c]
    \column{.5\linewidth}
    Co-Simulation within Paraview
    \begin{itemize}
    \item support for Catalyst
    \item directly run simulation with visualisation capabilities (in-situ)
    \end{itemize}

    Paraview plugin for Feel++
    \begin{itemize}
    \item directly integrate Feel++ code into paraview as a plugin
    \end{itemize}
    \column{.5\linewidth}
    %image here
  \end{columns}

\end{frame}

\section{Conclusion}

\begin{frame}{Summary}

  \begin{itemize}
  \item Lots of exciting (new) projects with a strong mutualisation of
    development
  \item The challenges for Feel++ are vertical (HPC, scaling to
    tens/hundrends of millions) and horizontal
    (MultiPhysics)
  \item The brainstorming would help identify recurrent issues,
    necessary/usefull developments...
  \end{itemize}

  Get the source of Feel++

  \begin{center}\url{http://github.com/feelpp/feelpp}\end{center}

\end{frame}

%%% Local Variables:
%%% mode: latex
%%% TeX-master: "lecture.feelpp.slides"
%%% End:
