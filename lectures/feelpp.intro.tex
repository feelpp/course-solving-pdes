\section{Introduction to Feel++}

\begin{frame}{Versatile}

  \begin{columns}[c]
    \column{.6\linewidth}
    \clipimg{image=figures/feelpp/Gallery-feelpp-600x600.jpg,image width=2cm}
    \column{.4\linewidth}
    A large range of \alert{numerical methods} to solve partial differential
    equations: cG, dG, hdG, crb, \ldots in 1D, 2D and 3D
  \end{columns}
\end{frame}

\begin{frame}{Powerful}

  \begin{columns}[c]
    \column{.4\linewidth}
    Support for \alert{high performance computing} up to thousands of cores
    for linear and non-linear problems using  \alert{PETSc/SLEPc} and
    \alert{InHouse} solution strategies
    \column{.6\linewidth}
    \centerline{\clipimg{image=figures/feelpp/supercomputer-1-600x600.jpg,image width=2cm}}
  \end{columns}
\end{frame}

\begin{frame}{Expressive}

  \begin{columns}[c]
        \column{.6\linewidth}

        \centerline{\clipimg{image=figures/feelpp/feelpp-dsel-600x600.png,image width=2cm}}

    \column{.4\linewidth}
    A language for \alert{Galerkin methods} embedded into C++ for
    maximal mathematical \alert{expressivity}.

    Support for meshes, function
    spaces and elements, bilinear and linear forms, algebraic
    representation and post-processing.
  \end{columns}
\end{frame}

\begin{frame}{Toolboxes :: Monophysics}
  \begin{columns}[c]
    \column{.3\linewidth}
    \centerline{ \Large CFD}\\[.5cm]
    \includegraphics[width=\linewidth]{figures/feelpp/400x200/FlowAroundCylinder-400x200.png}
    \column{.3\linewidth}
    \centerline{ \Large CSM}\\[.5cm]
    \includegraphics[width=\linewidth]{figures/feelpp/400x200/torsionbarNeoHookIncompT2-400x200.png}
    \column{.3\linewidth}
    \centerline{\Large Heat Transfer}\\[.5cm]
    \includegraphics[width=\linewidth]{figures/feelpp/400x200/heat-transfer-building-400x200.png}
  \end{columns}
\end{frame}

\begin{frame}{Toolboxes :: Multiphysics}
  \begin{columns}[c]
    \column{.3\linewidth}
    \centerline{\Large FSI}\\[.5cm]
    \includegraphics[width=\linewidth]{figures/feelpp/400x200/wp3dP3P2G2-struct-disp-t2-400x200.png}
    \column{.3\linewidth}
    \centerline{\Large Heat \& Fluid}\\[.5cm]
    \includegraphics[width=\linewidth]{figures/feelpp/400x200/cabine-400x200.png}
    \column{.3\linewidth}
    \centerline{\Large Thermoelectric}\\[.5cm]
    \includegraphics[width=\linewidth]{figures/feelpp/400x200/peltiermodule-electricpotential-400x200.png}
  \end{columns}
\end{frame}

\begin{frame}{A wide range of numerical methods}
  \begin{center}
  \centerline{\includegraphics[width=1.1\linewidth]{figures/feelpp/feelpp-methods.png}}
  \end{center}

\end{frame}

% \begin{frame}{\feel}
%   \begin{alertblock}{Finite Element Embedded Library and Language in C++}
%     A \alert{Domain Specific Language} for PDEs embedded in C++ providing a syntax very
%     close to the mathematical language
%   \end{alertblock}

%   \begin{itemize}
%   \item Supports \alert{simplex}, \alert{hypercube}, \alert{high
%       order} meshes as well as curves and surfaces
%   \item Supports generalized \alert{arbitrary order Galerkin methods
%       (cG, dG, hdG, crb, \ldots) in 1D, 2D and 3D}
%   \item Supports \alert{seamless interpolation}
%   \item Supports \alert{seamless parallel computing}
%   \item Supports \alert{large scale parallel linear and non-linear
%       solvers} (PETSc/SLEPc \& \alert{InHouse})
%   \item Supports various toolboxes : CFD, CSM, FSI, Heat Transfer, ThermoElectric
%   \end{itemize}
% \end{frame}

\begin{frame}{\feel}
  \begin{columns}[c]
    \column{.45\linewidth}
    \scalebox{.7}{
    \tikzstyle{root concept}+=[concept color=white!80]
    \tikzstyle{level 1 concept}+=[concept color=ljk!80, sibling angle=60,text=black]
    \tikzstyle{every annotation}=[fill=black!50,opacity=0.5,text=white,scale=.7]
    \begin{tikzpicture}[->]
      \path[mindmap,concept color=black!60,text=white]
      node[concept,scale=.7] {Feel++}
      % node[concept,scale=.7] {Scientific Computing}
      [clockwise from=0]
      % child[concept,scale=.6] { node[concept,scale=.7] (phys) {Physics Mechanics Biology Processing} }
    child[concept,scale=.6] { node[concept,scale=.7] (phys) {VIVABRAIN} }
    child[concept,scale=.6] { node[concept,scale=.7] (am) {HiFiMagnet} }
    child[concept,scale=.6] { node[concept,scale=.7] (nm) {HAMM / RB4FASTSIM} }
    child[concept,scale=.6] { node[concept,scale=.7] (cs) {CHORUS} }
    child[concept,scale=.6] { node[concept,scale=.7] (cs) {EYE2BRAIN} }
    child[concept,scale=.6] { node[concept,scale=.7] (cs) {Cemosis } }
    child[concept,scale=.6] { node[concept,scale=.7] (cs) {Teaching} }
    ;
    % \begin{pgfonlayer}{background}
    %   \draw [concept connection]
    %   (phys) edge node[above,sloped]{} (am)
    %   (am) edge node[above,sloped]{} (nm)
    %   (nm) edge node[above,sloped]{} (cs)
    %   (cs) edge node[above,sloped]{} (va)
    %   (va) edge node[above,sloped]{} (op);
    %   %(va) edge node[above,sloped]{} (phys);
    % \end{pgfonlayer}
   \end{tikzpicture}
   }
    \column{.45\linewidth}
    \scalebox{.7}{
\tikzstyle{root concept}+=[concept color=white!80]
\tikzstyle{level 1 concept}+=[concept color=ljk!80, sibling angle=90]
\tikzstyle{every annotation}=[fill=black!50,opacity=0.5,text=black]
\begin{tikzpicture}[mindmap,concept color=black!60,text=black]
  %%\path[clip] (-6,0) rectangle (6,6);
  \node[concept,text=white] (root)  {Complxity of scientific computing software}
  [clockwise from=45]
  child[concept] { node[concept] (mod) {Physical models} }
  child[concept] { node[concept] (alg) {Algebraic solvers} }
  child[concept] { node[concept] (cs) {Computer science} }
  child[concept] { node[concept] (num) {Numerical methods} };

  \node[name=dsl,fill=ljk!50,ellipse,decorate,decoration=snake,minimum width=3cm,
  minimum height=2cm] at (root.center) {\bf domain specific language for
  Galerkin methods};

  \node[name=a1,single arrow, draw=white!80, rotate=-90,scale=.2,fill=red] at
  (dsl.north) {};
  \node[name=a2,single arrow, draw, white!80, rotate=-90,scale=.2,fill=red] at
  (dsl.south) {};
  \node[decorate,decoration=snake,fill=ljk!50,above,ellipse,minimum
  height=1cm] at (a1.west)  {Express};
  \node[decorate,decoration=snake,fill=ljk!50,below,ellipse,minimum height=1cm] at (a2.tip)  {Generate};
%  \node[single arrow, draw, gray!50, rotate=-45,scale=.2,fill=black!60] at
%  (dsl.south east) {Generate};


  \begin{pgfonlayer}{background}

    %\draw[pink,thick] (root.north east) -- (root.north west);

    \draw [concept connection,white] (num.west) --
    node[fill=ljk!50,above,sloped,text=black,ellipse,minimum height=1cm]
    {Best expressivity with high level language}
    (mod.east) ;

    \draw [concept connection,white]
    (cs.west) -- node[fill=ljk!50,below,sloped,text=black,ellipse,minimum
    height=1cm] {Best performance with low level language} (alg.east) ;

    %\draw[concept connection,-,white] (root.north west) -- (root.north east);

  \end{pgfonlayer}



  % \path[fill=ljk!20,-]
  % (root.north west)  .. controls (0,1cm) .. (root.north east) -- (root.south east) .. controls (0,-1cm) .. (root.south west) --  (root.north west);
\end{tikzpicture}}

    %\includegraphics[width=\linewidth]{genprog2-crop.pdf}

  \end{columns}
  \begin{alertblock}{}
    Mathematical language for scientific computing
    \begin{itemize}
    \item to communicate between disciplines (Math, CS, Physics, Engineering...)
    \item to break complexity
    \end{itemize}
    \centerline{Usage Scenarii: Research, R\&D (``Bureau d'étude''), Teaching }
  \end{alertblock}
\end{frame}

\subsection{Partners and Collaborators}
\begin{frame}
  \frametitle{Partners/Collaborators}

  \begin{columns}[c]
    \column{.5\linewidth}
  \textbf{France}
  \begin{itemize}
  \item U. Strasbourg
  \item U. Grenoble Alpes
  \item UPMC
  \item LNCMI/CNRS
  \item U. Toulouse
  \end{itemize}
  \column{.5\linewidth}
  \textbf{Abroad}
  \begin{itemize}
  \item IUPUI (USA)
  \item IMATI (Italy)
  \item U. Coimbra (Portugal)
  \end{itemize}

  \end{columns}
\end{frame}

\begin{frame}{Companies}
  \begin{center}
    \centerline{\includegraphics[width=1\linewidth]{figures/feelpp/feelpp-companies.png}}
  \end{center}
\end{frame}
% \begin{frame}
%   \frametitle{People}

%   \begin{columns}[c]
%     \column{.5\linewidth}
%     \textbf{In}
%     \begin{itemize}
%   \item R. Hild (Eng.)
%   \item J.B. Wahl (Phd)
%   \item M. Spreng, T. Lantz,B. Vanthong (Msc)
%   \item M. Boileau (Eng CNRS)
%   \item G. Ghigliotti (MDC, LEGI, Grenoble) + Msc ?
%   \end{itemize}
%   \column{.5\linewidth}
%   \textbf{Out}
%   \begin{itemize}
%   \item V. Doyeux (Austin USA)
%   \item P. Jolivet (INF ETHZ)
%   \item A. Samake (Ice and See Lab, Norway)
%   \item S. Veys (CEA Saclay)
%   \item S. Priem, J. Veysset
%   \end{itemize}
%   \end{columns}
% \end{frame}
\subsection{Projects with Feel++}
\begin{frame}{Projects with Feel++}
  \begin{figure}[H]
    \centering
    \begin{subfigure}[b]{.3\linewidth}
      \includegraphics[width=1.\linewidth]{figures/applications/vivabrain.png}
    \end{subfigure}
    \begin{subfigure}[b]{.3\linewidth}
      \includegraphics[width=1.\linewidth]{figures/applications/eye2brain.png}
    \end{subfigure}
    \begin{subfigure}[b]{.3\linewidth}
      \includegraphics[width=1.\linewidth]{figures/applications/blood-rheology.png}
    \end{subfigure}\\
    \begin{subfigure}[b]{.3\linewidth}
      \includegraphics[width=1.\linewidth]{figures/applications/hemotumpp.png}
    \end{subfigure}
    \begin{subfigure}[b]{.3\linewidth}
      \includegraphics[width=1.\linewidth]{figures/applications/optical-tomography.png}
    \end{subfigure}
    \begin{subfigure}[b]{.3\linewidth}
      \includegraphics[width=1.\linewidth]{figures/applications/hifimagnet.png}
    \end{subfigure}

  \end{figure}

\end{frame}


\begin{frame}{Projects with Feel++}
  \begin{figure}[H]
    \centering
    \begin{subfigure}[b]{.3\linewidth}
      \includegraphics[width=1.\linewidth]{figures/applications/chorus.png}
    \end{subfigure}
    \begin{subfigure}[b]{.3\linewidth}
      \includegraphics[width=1.\linewidth]{figures/applications/holo3.png}
    \end{subfigure}
    \begin{subfigure}[b]{.3\linewidth}
      \includegraphics[width=1.\linewidth]{figures/applications/mso4sc.png}
    \end{subfigure}\\
    \begin{subfigure}[b]{.3\linewidth}
      \includegraphics[width=1.\linewidth]{figures/applications/po.png}
    \end{subfigure}
    \begin{subfigure}[b]{.3\linewidth}
      \includegraphics[width=1.\linewidth]{figures/applications/bioreactor-sivibirpp.png}
    \end{subfigure}
    \begin{subfigure}[b]{.3\linewidth}
      \includegraphics[width=1.\linewidth]{figures/applications/gazomat.png}
    \end{subfigure}

  \end{figure}

\end{frame}


\section{Feel++ on the web}


\begin{frame}{\feel web ecosystem}
  \begin{itemize}
  \item \faChrome: \url{http://www.feelpp.org}
  \item \faGithub :: \url{https://github.com/feelpp/feelpp}
  \item \faGooglePlus: \url{http://plus.feelpp.org}
  \item \faForumbee: \url{http://forum.feelpp.org} (Gitter)
  \item \faNewspaperO: \url{http://publications.feelpp.org} (Hal)
  \item \faYoutube : \url{http://youtube.feelpp.org}
  \item \faTwitter: \url{http://twitter.feelpp.org}
  \end{itemize}
\end{frame}



\section{Installing Feel++}

\subsection{Docker (recommended)}

\begin{frame}{Docker (recommended)}
  This is the preferred way for end users


\end{frame}

\subsection{Linux}

\begin{frame}{Ubuntu}

\end{frame}

\begin{frame}{Debian}

\end{frame}

\subsection{MacOS X}


\begin{frame}[fragile]{Using Homebrew}
  \begin{bashcode}
    brew tap
  \end{bashcode}
\end{frame}


\subsection{Windows?}


%%% Local Variables:
%%% mode: latex
%%% TeX-master: "lecture.feelpp.slides"
%%% End:
